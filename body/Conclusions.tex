\chapter{Conclusions}

With COVID-19 bringing population prediction into the public consciousness, the need for an understanding of how this could be implemented was manifest. The success of this project, as presented in its brief, hinged on its ability to derive meaningful information about the process of app development, and its ability to gather insight into smartphones' capability to act as crowd-sourced location data sources. To begin the project's investigation, the literature surrounding the project objective was analysed, with existing services used as reference for how the project should approach its objective. This research assisted the formulation of the project's strategic approach, objectives, requirements, and specifications which would be used to guide the development process. 

The design phase commenced with a discussion about the possible approaches towards important project design concepts. After the chosen approach was decided upon, the systems constituting the PEEPS framework were designed. The implementation phase then began with the creation of a dedicated database and web interface. Once the web interface was able to manipulate the database's data correctly, development on the PEEPS app was started. The completion of the app marked the beginning of this project's testing phase.

The results gathered from the PEEPS application's functional and investigative testing are promising for developers wishing to implement crowd-sourced location functionality, using the PEEPS framwork, as the app was able to provide the user with meaningful, actionable, and relevant data that was both relatively certain and timely. The functional testing verified that the app was able to gather useful data from the database and present it to the user. 

All three investigative tests provided useful insights into the operation of this device. The location uncertainty test, Test-3, determined that there is a standard deviation of 3.0525 meters from the mean when uploading a device's location. Since 68\% of recorded values are calculated to be within one standard deviation of the mean, we can expect an uncertainty of less than three meters for most readings. Similarly, we can expect an variation of 6.1050 meters for 95\% of recordings. However, these values can be decreased by increasing the number of recordings taken, at the cost of processing time.

Test-4, which tested the consistency of Android's job manager, found that the app's location upload JobService ran almost exactly as specified. This is promising for developers wanting an upload schedule with a consistent upload interval within seconds of what the program specified. Test-5 also displayed the app's reliability determining that the app's resource usage was minimal and should run on any compatible device.

Consequent to the testing, the list of specifications was verified, and a discussion was conducted about whether the project's implementation was successful. Finally, conclusions were drawn, and recommendations were given based on the end-product delivered by the project. 
