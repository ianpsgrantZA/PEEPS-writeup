\chapter{Introduction}

\section{Background to the study}
With the advent of the novel Coronavirus appearing towards the end of 2019, population prediction mechanisms have suddenly become commonplace in the minds of the public. With a virus that is highly infectious in locations with high population density, it is in the interest of the public’s health to minimise excess contact with potentially infected persons. While this pandemic has brought the health benefits of population evaluation to the public consciousness, the potential for population prediction to improve personal and general prosperity has long been available. It is as important for consumers to determine which shopping centres have shorter lines as it is for a new business owner to know what is causing the traffic levels in their store. Hence, both population evaluation and population prediction have the potential to revolutionise how we predict prosperity and enhance our own lifestyle. 

Large scale population data tracking has become increasingly viable in the last decade due to the precise positioning sensors on smartphones and their extensive prevalence. Population evaluation mechanisms no longer require third party cameras and beacons, but instead favour crowd-sourced data uploaded from mobile devices. The most successful example of this kind of mass population evaluation and prediction is Google’s Maps smartphone app. Maps uses mass user-data to predict live population counts of roads, businesses, and public locations. This type of population tracking does however come with a healthy amount of concern for privacy. With the great benefits of population prediction comes the responsibility for preventing bad actors from abusing this information to commit malicious activities. Hence, population evaluation systems need to focus their efforts on increasing public trust while simultaneously solving problems using this powerful tool.

\section{Objectives of this study}
Due to the immense benefit of population evaluation and the ability for it to be integrated into many applications, this study aims to develop a modular framework with design aspects that can be incorporated into new and existing projects. This framework will take the shape of a smartphone application with integrated network and database connectivity. The app will implement various mechanisms of population tracking, evaluation and prediction which will be complemented by a robust server system. 

This project will provide documentation on the design process for these population evaluation mechanisms and conduct testing on the functionality of the system. Additionally, investigative research will be conducted on the system to provide future developers with insight into the operation of such a population evaluation tool. 
This project emphasises modularity and system-focussed design to allow for future developers to choose which aspects of population evaluation they wish to use. The program will be documented thoroughly for the same purpose.


\section{Scope and Limitations}
The scope of this project is limited to the design and research testing of a typical population evaluation tool. So, no large-scale population testing will be conducted, but instead the individual systems of the PEEPS Framework will be evaluated to determine their functional parameters. Likewise, the application will be tested using emulators and one typical modern smartphone and as such, may not be representative of how the system will react to all devices. 

This project will also be limited to the resources available at the height of the COVID-19 pandemic. This means laboratory hardware and resources will not be utilised nor will testing on non-research individuals be conducted.
Since the PEEPS system’s objective is for it to be used as a framework, this creates a difficulty for it to become a publish-ready application, as the program’s focus will be on creating systems that can be incorporated by other projects, rather than being a completely fleshed out application.


\section{Plan of development}
The project was initiated by performing a thorough analysis on the landscape of app development and population prediction mechanisms. This review was then used to formulate the design concepts outlined in the methodology. These design concepts were then finalised and development on the PEEPS server system commenced. This included construction of a robust database and web interface. Once construction of the server was at a functional standard, development on the smartphone app begun. The development on the app employed the use of a smartphone emulator to allow for swift and concurrent development and functional testing. After the finalisation of the PEEPS Framework, functional and research tests were devised and conducted to determine the operating characteristics on the completed system. Finally, conclusions were drawn based on the results of the tests and recommendations were formulated.

\section{Plan of development}
TODO
