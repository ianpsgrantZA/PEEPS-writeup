%\addcontentsline{toc}{section}{Abstract}
{\Large Abstract}\\
\hrule

Recognising the pressing need for an exploration of population density prediction mechanisms brought to light by COVID-19, this project aims to investigate the design and implementation of crowd-sourced population evaluation mechanisms. This project uses a comprehensive review of the app development, location tracking, and data acquisition literature in order to develop a robust population tracking smartphone application framework, named the Population of Environment Evaluation and Prediction System (PEEPS). This application implements user recognition, location sharing, and data processing to provide users with actionable information about the density of localised populations. Additionally, the smartphone app provides the user with insight into how many people they are likely to encounter throughout the course of the day. PEEPS is used as a mechanism to investigate the development of a crowd-sourced and location-predictive application. Additionally, the application undergoes both functional and investigative testing to determine its abilities and shortcomings. This project intends to inform developers who wish to implement crowd-sourced location prediction about the design process and implementation of such an application, while also providing such developers with the necessary insights and codebase to jumpstart development. The testing that was conducted on the application proved the app's ability to provide potential users with information that is accurate, timely and actionable. In addition, the testing determined that the app used insubstantial smartphone resources, thus proving its ability to act as a scalable framework for developers to utilise in their own location prediction applications. Lastly, the project provides recommendations gathered from the investigative testing and presents opportunities for future development.